%! Author = Zhanka
%! Date = 12-Jun-22

% Preamble
\documentclass[11pt]{report}

% Packages
\usepackage{amsmath}
\usepackage{graphicx}
\usepackage{tabularx}
\usepackage[margin=1.2in]{geometry}

% Document
\begin{document}

    \begin{titlepage}
        \vfill
        \thispagestyle{empty}
        \setcounter{page}{1}

        \begin{figure}[t]
            \centering
            \includegraphics[height=4cm]{./img/logo.png}
        \end{figure}

        \begin{center}
            \Large{\bfseries\scshape{SISWATI BIBLE APP}} \\
            \scshape{\small{with licilogo}} \\
            \line(1, 0){\textwidth} \\
            \vspace{3.75cm}

            % \scshape{\large{}}
            \scshape{\Large{{{A MODERN MOBILE APPLICATION(APP) \\[0.5cm]FOR
            \\[0.5cm]ACCESSING SISWATI BIBLE AND LICILONGO\\[0.5cm] OFFLINE}}}}  \\

            \vspace{0.7cm}

            \vfill
            \vspace{0.3cm}
            \begin{flushright}
                \normalsize{\textbf{Prepared By} }: \scshape{\normalsize{Lindelwa Sifiso Dlamini}}\\
                \normalsize{\textbf{Cell} }: \scshape{\normalsize{+268 7648 0479}}
            \end{flushright}

            %\vspace{0.1cm}
            \line(1, 0){\textwidth} \\
            \scshape{\normalsize{\today}}

        \end{center}

    \end{titlepage}

    \begin{center}
        \begin{figure}[h!]
            \begin{minipage}[h!]{0.30\textwidth}
                \centering
                \fbox{\includegraphics[width=\textwidth, height=0.8\textheight, keepaspectratio]{./img/home_nn.jpg}}
                \caption{Home}
            \end{minipage}
            \hfill
            \begin{minipage}[h!]{0.3\textwidth}
                \centering
                \fbox{\includegraphics[width=\textwidth, height=0.8\textheight, keepaspectratio]{./img/home_n.jpg}}
                \caption{Home-Dark}
            \end{minipage}
            \hfill
            \begin{minipage}[h!]{0.3\textwidth}
                \centering
                \fbox{\includegraphics[width=\textwidth, height=0.8\textheight, keepaspectratio]{./img/verses_nn.jpg}}
                \caption{Verses}
            \end{minipage}
        \end{figure}
    \end{center}

    \section*{Executive Summary}\label{sec:executive-summary}
    Technology has integrated itself seamlessly into our daily activities, improving and sustaining them.
    There is no denying that mobile apps have changed our lives in every way.
    There is an app for almost everything, whether you need to travel, shop online, order food, or conduct banking
    transactions.
    Numerous church apps are also available, but according to my research, only a few of these apps specifically
    target emaSwati.
    Only a few, poorly engineered apps are available in the Siswati translation.
    This calls for the development of a mobile app that is specifically designed to accommodate emaSwati and
    should include a variety of features to help our brothers and sisters be productive while also providing 24 hour
    access to the Holy Bible, irregardless of where the person is.\\

    This document entails a project that was initiated to develop a mobile app for Android devices that allows users
    to access the Siswati Bible strictly offline.
    It also has access to Licilongo and other features such as access to multiple public domain bible translations
    (ASV, KJV, etc\ldots), allowing the user to choose different translations based on their preferences.
    All of these features are rendered offline, and the user does not require mobile data.\\

    The idea came to me when I was in need of a mobile app with the Siswati version.
    I couldn't find a modern bible app with the Siswati translation after hours of searching online.
    The only available resource was a very old app called ``Siswati Bible", which was credited to Sandile Mkhaliphi.
    This project is really old and does not fully support modern versions of Android.\\

    This then sparked the idea to create a modernised free bible with the Siswati translation.
    I've spent over 2 years developing the entire application to make sure it is up-to-standard with recent
    programming practices.
    For now the app is available for Android users.
    See end of document for screen-shots of some of the features available.
    Below I list some of its features.\\

    \section*{App features}
    The application has several useful features to make reading the bible much more enjoyable and a good experience.
    \begin{enumerate}
        \item The application is designed to function entirely offline.
        You do not need to be online to access the bible, which is beneficial for users in remote and rural areas where
        internet access is not available.
        \item The application comes with a full copy of Licilongo.
        This makes it easier for people that often use it.
        This feature removes the need of having a different app or the hard copy for Licilongo.
        \item The application also comes with a note feature to allow users to write down important notes for
        themselves.
        \item The application also comes with a search functionality to quickly navigate to desired books and chapters.
        \item App also supports dark mode.
        For eye sensitive users the application comes with dark mode features.
        \item The app also allows users to save and share screenshots of their favorite verses, which they can then
        share on social media platforms.
        Our logo is embedded on these screenshots to spread word about the app.
        \item Has a modern and intuitive user interface.
        \item App is less than 10MBs and once installed takes less than 20MBs thus saving user device storage.
        \item A currently experimental feature allows users to download other translations such as KJV or SIZULU, and
        then translate back and forth or select the default version.
        This is beneficial for users who prefer English over Siswati.
        The following versions are currently available: SISWATI, NIV, SIZULU, ASV, YLT, WBT, BBE, and KJV\@.
        To avoid copyright issues, downloadable versions are public domain versions (free).
    \end{enumerate}

    \section*{How to monetize the application}
    Although the app is free and should be distributed as such, I have a few ideas for monetizing it.
    Following my research, I discovered an online shop for the Bible Society of Eswatini, which can easily be integrated
    into the application to allow users to access the shop.
    MoMo Pay could also be used to make transactions easier.
    As useful as the app is, users will still require a hard copy of the Bible, and this shop will make it easier for
    users to access and browse available hard copy bibles.\\

    The application can also be used to advertise and sell any of the Society's products (any merchandise offered)\@.
    For example, I saw a Facebook post about a prayer recording that was made available through WhatsApp;
    with the introduction of the app, it would be possible to sell the recordings directly through the app;
    once a user has paid via mobile money, they have instant access to it and can also play it via the app
    (this is much more convenient than manually sending via WhatsApp)\@.\\

    The app can also be redesigned for marketing purposes to include your logos and contact information.\\

    Some of the listed features may be removed and offered as premium features, for example, Licilongo may be available
    only to users who have subscribed and paid a fee.\\

    The app can also be used to access the Society's online bible courses.
    This provides a more convenient platform for accessing such courses than using a browser because everything will
    be customized and the design will be much simpler, resulting in less mobile data usage for the user.
    Authentication will, of course, be required to ensure that only authorized users have access.\\

    The app can also be used for advertising;
    other churches can use it to promote their services or products, and we can charge them for it.

    \section*{My Expectations}
    I created this app solely to spread God's message and gain experience in developing real-world practical software.
    My only expectation is to be able to distribute the app through the Google Play Store, which will be useful for me
    as a developer because this will keep track of downloads.
    I should also be given credit for creating the application so that I can list it as one of my accomplishments on my
    resume.
    Aside from that, you can re-design the app and add your logos for marketing purposes.
    My only wish is that you assist me in getting the app published.

    \section*{About Me}
    I'm a final-year Electrical and Electronics Engineering student at the University of Eswatini.
    I'm an aspiring software developer with at least three years of experience.
    Working on this project has been both enjoyable and educational for me.
    To make this project a success, I used cutting-edge tools and programming techniques, and so with this project, I
    hope to spread the word of God to all Swazis.
    I believe it will significantly improve church organization and provide constant access to the Holy Bible.

    For more details about me or the project, use the following contacts details to contact me.
    \begin{center}
        \begin{figure}[h]
            \centering
            \renewcommand{\arraystretch}{1.5}%
            \begin{tabularx}{\textwidth}{|X|X|}
                \hline\textbf{Full name} & Dlamini Lindelwa Sifiso         \\
                \hline\textbf{Cell}      & +268 7648 0479 / +268 7948 0479 \\
                \hline\textbf{Email}     & sfisolindelwa@gmail.com         \\\hline
            \end{tabularx}
        \end{figure}
    \end{center}

    \section*{App Show Case}\label{sec:showcase}
    \input{app-show-case}

\end{document}